\documentclass{report}
\usepackage{tabularx} % extra features for tabular environment
\usepackage{amsmath}  % improve math presentation
\usepackage{graphicx} % takes care of graphic including machinery
\usepackage[margin=1in,letterpaper]{geometry} % decreases margins
\usepackage{cite} % takes care of citations
\usepackage[final]{hyperref} % adds hyper links inside the generated pdf file
\usepackage[utf8]{inputenc}

\graphicspath{ {./images/} }

\title{MCE - Movimento de Projéteis}
\author{\texttt{102480} - Rúben Pequeno
\and 
\texttt{nmec} - Filipe Sousa
\and
\texttt{932779} - António Moreira}
\date{\today}
\renewcommand*\contentsname{Conteúdos}

\begin{document}

\maketitle

\tableofcontents
\newpage

\chapter{Introdução}

\section{Objetivos}
	Este trabalho tem como objetivo verificar o comportamento de diversos tipos de movimento de projéteis tais como 
	\renewcommand{\theenumi}{\Alph{enumi}}
	\begin{enumerate}
		\item{Determinar a velocidade inicial do projétil e calcular o respetivo erro}
		\item{Verificar a dependência do alcance com o ângulo de lançamento}
		\item{Determinar a velocidade inicial do projétil utilizando um pêndulo balístico}
	\end{enumerate}


\section{Resultados Esperados vs Resultados Obtidos}
	Comparando os dados obtidos com os dados esperados, podemos observar que apesar de algumas
	diferenças nos valores devido a possiveis erros de medição e de execução experimental, verifica-se
	que o comportamento do projétil está de acordo com o resultado esperado.
	

\chapter{Parte A}

\section{Preparação}
	Para determinar-mos a velocidade inicial do projétil, tivemos de fazer uma montagem experimental
	 utilizando parte dos materiais fornecidos (lançador de projéteis, sensores de passagem e a esfera plástica) 
	 
	 e coloca-los tal como está descrito Figura 1.


\begin{figure}
	\centering
	\includegraphics{part_a.png}
	\caption{Esquema da montagem experimental (experiência A). 1- Lançador de projéteis (LP); 2-
	Base de fixação para o LP; 3-Sensor de passagem (inicia a contagem do tempo); 4-Sensor de
	passagem (termina a contagem do tempo); 5-Sistema de controlo dos sensores.}
	\label{fig:parteA}
\end{figure}

\section{Dados}

\section{Erros}

\chapter{Parte B}

\section{Preparação}

	Para esta experiência foi aproveitado a montagem feita anteriormente, apenas tirando
	os sensores de passagem, pois já não será necessário calcular a velocidade do projétil 
	e adicionando um alvo (sensor de impacto), de modo a podermos medir a distância percorrida 
	pelo projétil, a um determinado angulo.

\begin{figure}
	\centering
	\includegraphics{part_b.png}
	\caption{Esquema da montagem experimental (experiência B). 1-Lançador de projéteis (LP); 2-
	Base de fixação para o LP; 3-Alvo.}
\end{figure}

\section{Dados}

\section{Erros}

\chapter{Parte C}

\section{Preparação}
\begin{figure}
	\centering
	\includegraphics{part_c.png}
	\caption{Esquema da montagem experimental (experiência C).}
\end{figure}

\section{Dados}

\section{Erros}

\chapter{Conclusões}
Conclusões.

\end{document}
