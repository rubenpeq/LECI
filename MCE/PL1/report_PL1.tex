\documentclass{report}
\usepackage{tabularx} % extra features for tabular environment
\usepackage{amsmath}  % improve math presentation
\usepackage{graphicx} % takes care of graphic including machinery
\usepackage[margin=1in,letterpaper]{geometry} % decreases margins
\usepackage{cite} % takes care of citations
\usepackage[final]{hyperref} % adds hyper links inside the generated pdf file

\title{MCE - Movimento de Projéteis}
\author{\texttt{102480} - Rúben Pequeno
\and 
\texttt{nmec} - Filipe
\and
\texttt{nmec} - António}
\date{\today}
\renewcommand*\contentsname{Conteúdos}

\begin{document}

\maketitle

\tableofcontents
\newpage

\chapter{Introdução}

\section{Objetivos}
	Este trabalho tem como objetivo verificar o comportamento de diversos tipos de movimento de projéteis tais como 
	\renewcommand{\theenumi}{\Alph{enumi}}
	\begin{enumerate}
		\item{Determinar a velocidade inicial do projétil através das equações do movimento}
		\item{Verificar a dependência do alcance com o ângulo de lançamento}
		\item{Determinar a velocidade inicial do projétil utilizando um pêndulo balístico}
	\end{enumerate}


\section{Resultados Esperados vs Resultados Obtidos}
	

\chapter{Teoria}

\section{Parte A}

\section{Parte B}

\section{Parte C}

\chapter{Procedimento Exprimental}

\section{Parte A}

\section{Parte B}

\section{Parte C}

\chapter{Resultados}

\section{Parte A}

\section{Parte B}

\section{Parte C}

\chapter{Conclusions}
Conclusões.

\end{document}
